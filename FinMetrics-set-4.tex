% Options for packages loaded elsewhere
\PassOptionsToPackage{unicode}{hyperref}
\PassOptionsToPackage{hyphens}{url}
%
\documentclass[
]{article}
\usepackage{amsmath,amssymb}
\usepackage{iftex}
\ifPDFTeX
  \usepackage[T1]{fontenc}
  \usepackage[utf8]{inputenc}
  \usepackage{textcomp} % provide euro and other symbols
\else % if luatex or xetex
  \usepackage{unicode-math} % this also loads fontspec
  \defaultfontfeatures{Scale=MatchLowercase}
  \defaultfontfeatures[\rmfamily]{Ligatures=TeX,Scale=1}
\fi
\usepackage{lmodern}
\ifPDFTeX\else
  % xetex/luatex font selection
\fi
% Use upquote if available, for straight quotes in verbatim environments
\IfFileExists{upquote.sty}{\usepackage{upquote}}{}
\IfFileExists{microtype.sty}{% use microtype if available
  \usepackage[]{microtype}
  \UseMicrotypeSet[protrusion]{basicmath} % disable protrusion for tt fonts
}{}
\makeatletter
\@ifundefined{KOMAClassName}{% if non-KOMA class
  \IfFileExists{parskip.sty}{%
    \usepackage{parskip}
  }{% else
    \setlength{\parindent}{0pt}
    \setlength{\parskip}{6pt plus 2pt minus 1pt}}
}{% if KOMA class
  \KOMAoptions{parskip=half}}
\makeatother
\usepackage{xcolor}
\usepackage[margin=1in]{geometry}
\usepackage{color}
\usepackage{fancyvrb}
\newcommand{\VerbBar}{|}
\newcommand{\VERB}{\Verb[commandchars=\\\{\}]}
\DefineVerbatimEnvironment{Highlighting}{Verbatim}{commandchars=\\\{\}}
% Add ',fontsize=\small' for more characters per line
\usepackage{framed}
\definecolor{shadecolor}{RGB}{248,248,248}
\newenvironment{Shaded}{\begin{snugshade}}{\end{snugshade}}
\newcommand{\AlertTok}[1]{\textcolor[rgb]{0.94,0.16,0.16}{#1}}
\newcommand{\AnnotationTok}[1]{\textcolor[rgb]{0.56,0.35,0.01}{\textbf{\textit{#1}}}}
\newcommand{\AttributeTok}[1]{\textcolor[rgb]{0.13,0.29,0.53}{#1}}
\newcommand{\BaseNTok}[1]{\textcolor[rgb]{0.00,0.00,0.81}{#1}}
\newcommand{\BuiltInTok}[1]{#1}
\newcommand{\CharTok}[1]{\textcolor[rgb]{0.31,0.60,0.02}{#1}}
\newcommand{\CommentTok}[1]{\textcolor[rgb]{0.56,0.35,0.01}{\textit{#1}}}
\newcommand{\CommentVarTok}[1]{\textcolor[rgb]{0.56,0.35,0.01}{\textbf{\textit{#1}}}}
\newcommand{\ConstantTok}[1]{\textcolor[rgb]{0.56,0.35,0.01}{#1}}
\newcommand{\ControlFlowTok}[1]{\textcolor[rgb]{0.13,0.29,0.53}{\textbf{#1}}}
\newcommand{\DataTypeTok}[1]{\textcolor[rgb]{0.13,0.29,0.53}{#1}}
\newcommand{\DecValTok}[1]{\textcolor[rgb]{0.00,0.00,0.81}{#1}}
\newcommand{\DocumentationTok}[1]{\textcolor[rgb]{0.56,0.35,0.01}{\textbf{\textit{#1}}}}
\newcommand{\ErrorTok}[1]{\textcolor[rgb]{0.64,0.00,0.00}{\textbf{#1}}}
\newcommand{\ExtensionTok}[1]{#1}
\newcommand{\FloatTok}[1]{\textcolor[rgb]{0.00,0.00,0.81}{#1}}
\newcommand{\FunctionTok}[1]{\textcolor[rgb]{0.13,0.29,0.53}{\textbf{#1}}}
\newcommand{\ImportTok}[1]{#1}
\newcommand{\InformationTok}[1]{\textcolor[rgb]{0.56,0.35,0.01}{\textbf{\textit{#1}}}}
\newcommand{\KeywordTok}[1]{\textcolor[rgb]{0.13,0.29,0.53}{\textbf{#1}}}
\newcommand{\NormalTok}[1]{#1}
\newcommand{\OperatorTok}[1]{\textcolor[rgb]{0.81,0.36,0.00}{\textbf{#1}}}
\newcommand{\OtherTok}[1]{\textcolor[rgb]{0.56,0.35,0.01}{#1}}
\newcommand{\PreprocessorTok}[1]{\textcolor[rgb]{0.56,0.35,0.01}{\textit{#1}}}
\newcommand{\RegionMarkerTok}[1]{#1}
\newcommand{\SpecialCharTok}[1]{\textcolor[rgb]{0.81,0.36,0.00}{\textbf{#1}}}
\newcommand{\SpecialStringTok}[1]{\textcolor[rgb]{0.31,0.60,0.02}{#1}}
\newcommand{\StringTok}[1]{\textcolor[rgb]{0.31,0.60,0.02}{#1}}
\newcommand{\VariableTok}[1]{\textcolor[rgb]{0.00,0.00,0.00}{#1}}
\newcommand{\VerbatimStringTok}[1]{\textcolor[rgb]{0.31,0.60,0.02}{#1}}
\newcommand{\WarningTok}[1]{\textcolor[rgb]{0.56,0.35,0.01}{\textbf{\textit{#1}}}}
\usepackage{graphicx}
\makeatletter
\def\maxwidth{\ifdim\Gin@nat@width>\linewidth\linewidth\else\Gin@nat@width\fi}
\def\maxheight{\ifdim\Gin@nat@height>\textheight\textheight\else\Gin@nat@height\fi}
\makeatother
% Scale images if necessary, so that they will not overflow the page
% margins by default, and it is still possible to overwrite the defaults
% using explicit options in \includegraphics[width, height, ...]{}
\setkeys{Gin}{width=\maxwidth,height=\maxheight,keepaspectratio}
% Set default figure placement to htbp
\makeatletter
\def\fps@figure{htbp}
\makeatother
\setlength{\emergencystretch}{3em} % prevent overfull lines
\providecommand{\tightlist}{%
  \setlength{\itemsep}{0pt}\setlength{\parskip}{0pt}}
\setcounter{secnumdepth}{-\maxdimen} % remove section numbering
\ifLuaTeX
  \usepackage{selnolig}  % disable illegal ligatures
\fi
\usepackage{bookmark}
\IfFileExists{xurl.sty}{\usepackage{xurl}}{} % add URL line breaks if available
\urlstyle{same}
\hypersetup{
  pdftitle={Forecasting FinMetrics},
  pdfauthor={Oskar Allerslev},
  hidelinks,
  pdfcreator={LaTeX via pandoc}}

\title{Forecasting FinMetrics}
\author{Oskar Allerslev}
\date{2024-09-21}

\begin{document}
\maketitle

suppose that data follows a stationary AR(1) model on the form

\(x_t = \mu + \rho x_{t-1}+\epsilon_t, \hspace{0,35cm} t\geq 1\)

With \(\epsilon\) iid \(\mathcal{N}(0,\sigma^2), \sigma^2 > 0\).

\begin{enumerate}
\def\labelenumi{\arabic{enumi}.}
\tightlist
\item
  Provide an expression for \(E(X_{T+h} \mid \mathcal{F}_t)\) as a
  function of \(x_T\) and the parameters of the model. Also show
  \(x_{T+h} \mid X_{t} \overset{p}{\rightarrow} E(x_t)\) as
  \(h \rightarrow \infty\)
\end{enumerate}

\[
\begin{align*}
E(x_{T+h} \mid \mathcal{F}_T) &= E(\mu + \rho x_{T}+\epsilon_{T+h} \mid \mathcal{F}_T) \\
&=\mu + \rho x_{T} + E(\epsilon_{T+h})\\
&=\mu +\rho(\mu + \rho x_{T+1}) \\
&= \dots \\
&=\mu \sum_{i=0}^{h-1}\rho^i + \rho^h x_T
\end{align*}
\]

we utilize that \(\rho^h x_{T}\) tends to zero as
\(h \rightarrow \infty\) since \(\mid \rho \mid < 1\) given that data is
a stationary time series. \[
\begin{align*}
E(x_{T+h} \mid \mathcal{F}_T) =\mu \sum_{i=0}^{h-1} \rho^i= \mu\frac{1-\rho^h}{1-\rho} \overset{h \rightarrow \infty}{=} \frac{\mu}{1-\rho} = E(x_t^*) 
\end{align*}
\] for a stationary series.

\begin{enumerate}
\def\labelenumi{\arabic{enumi}.}
\setcounter{enumi}{1}
\tightlist
\item
  Make a simulation and plot the points \((t, x_t \mid x_{T})\) for
  appripriate values for the model parameters.
\end{enumerate}

\begin{Shaded}
\begin{Highlighting}[]
\FunctionTok{library}\NormalTok{(ggplot2)}

\NormalTok{mu }\OtherTok{\textless{}{-}} \FloatTok{0.2}
\NormalTok{rho }\OtherTok{\textless{}{-}} \FloatTok{0.9}
\NormalTok{sigma }\OtherTok{\textless{}{-}} \DecValTok{1}
\NormalTok{T }\OtherTok{\textless{}{-}} \DecValTok{1000}
\NormalTok{h }\OtherTok{\textless{}{-}} \DecValTok{100}


\NormalTok{x }\OtherTok{\textless{}{-}} \FunctionTok{numeric}\NormalTok{(T}\SpecialCharTok{+}\NormalTok{h)}
\NormalTok{x[}\DecValTok{1}\NormalTok{] }\OtherTok{\textless{}{-}} \DecValTok{0}

\FunctionTok{set.seed}\NormalTok{(}\DecValTok{1}\NormalTok{)}
\NormalTok{epsilon }\OtherTok{\textless{}{-}} \FunctionTok{rnorm}\NormalTok{(T}\SpecialCharTok{+}\NormalTok{h, }\AttributeTok{mean =} \DecValTok{0}\NormalTok{, }\AttributeTok{sd =}\NormalTok{ sigma)}
\ControlFlowTok{for}\NormalTok{ (t }\ControlFlowTok{in} \DecValTok{2}\SpecialCharTok{:}\NormalTok{(T}\SpecialCharTok{+}\NormalTok{h))\{}
\NormalTok{  x[t] }\OtherTok{\textless{}{-}}\NormalTok{ mu }\SpecialCharTok{+}\NormalTok{ rho }\SpecialCharTok{*}\NormalTok{x[t}\DecValTok{{-}1}\NormalTok{]}\SpecialCharTok{+}\NormalTok{epsilon[t]}
\NormalTok{\}}

\NormalTok{long\_term\_mean }\OtherTok{\textless{}{-}}\NormalTok{ mu}\SpecialCharTok{/}\NormalTok{(}\DecValTok{1}\SpecialCharTok{{-}}\NormalTok{rho)}

\NormalTok{data }\OtherTok{\textless{}{-}} \FunctionTok{data.frame}\NormalTok{(}\AttributeTok{Time =} \DecValTok{1}\SpecialCharTok{:}\NormalTok{(T}\SpecialCharTok{+}\NormalTok{h),}\AttributeTok{x =}\NormalTok{ x)}


\FunctionTok{ggplot}\NormalTok{(data, }\FunctionTok{aes}\NormalTok{(}\AttributeTok{x =}\NormalTok{ Time, }\AttributeTok{y =}\NormalTok{ x)) }\SpecialCharTok{+}
  \FunctionTok{geom\_line}\NormalTok{(}\AttributeTok{color =} \StringTok{"blue"}\NormalTok{) }\SpecialCharTok{+}
  \FunctionTok{geom\_hline}\NormalTok{(}\AttributeTok{yintercept =}\NormalTok{ long\_term\_mean, }\AttributeTok{linetype =} \StringTok{"dashed"}\NormalTok{, }\AttributeTok{color =} \StringTok{"red"}\NormalTok{) }\SpecialCharTok{+}
  \FunctionTok{labs}\NormalTok{(}\AttributeTok{title =} \StringTok{"Simulated AR(1) Series"}\NormalTok{,}
       \AttributeTok{x =} \StringTok{"Time (t)"}\NormalTok{, }\AttributeTok{y =} \StringTok{"x\_t"}\NormalTok{) }\SpecialCharTok{+}
  \FunctionTok{theme\_minimal}\NormalTok{() }\SpecialCharTok{+}
  \FunctionTok{annotate}\NormalTok{(}\StringTok{"text"}\NormalTok{, }\AttributeTok{x =}\NormalTok{ T }\SpecialCharTok{+} \DecValTok{5}\NormalTok{, }\AttributeTok{y =}\NormalTok{ long\_term\_mean, }\AttributeTok{label =} \FunctionTok{sprintf}\NormalTok{(}\StringTok{"Long{-}term Mean = \%.2f"}\NormalTok{, long\_term\_mean), }
           \AttributeTok{color =} \StringTok{"red"}\NormalTok{, }\AttributeTok{hjust =} \DecValTok{1}\NormalTok{)}
\end{Highlighting}
\end{Shaded}

\includegraphics{FinMetrics-set-4_files/figure-latex/unnamed-chunk-1-1.pdf}
which is consistent with our convergence result.

\begin{enumerate}
\def\labelenumi{\arabic{enumi}.}
\setcounter{enumi}{2}
\tightlist
\item
  Consider the forecast error
  \(\epsilon_{t+h\mid T} = x_{t+h}-x_{T+h\mid T}\). Show
\end{enumerate}

\[
\begin{align*}
\epsilon_{T+h\mid T} \sim \mathcal{N}(0,\sigma_h^2)
\end{align*}
\] We start by finding an expression for the above

\[
\begin{align*}
\epsilon_{t+h\mid T} &= x_{T+h}- x_{T+h\mid T}\\
&= \mu \sum_{i=0}^{h-1} \rho^i +\rho^h x_T +\sum_{j=0}^{h-1}\rho^j \epsilon_{T+h-j}-\left( \mu \sum_{i=0}^{h-1} \rho^i +\rho^h x_T \right)\\
&=\sum_{j=0}^{h-1} \rho^j \epsilon_{T+h-j}
\end{align*}
\] Which is a scaled sum of iid \(\mathcal{N}(0,\sigma^2)\). Hence \[
\begin{align*}
E(\epsilon_{t+h\mid T})&=E\left( \sum_{j=0}^{n-1} \rho^{j} \epsilon_{T+h-j} \right)=0\\
E(\epsilon_{t+h\mid T})&=E\left( \sum_{j=0}^{n-1} \rho^{2j}\epsilon_{T+h-j}^2 \right)=\sigma^2 \sum_{j=0}^{h-1}\rho^{2j}=\sigma^2 \frac{1-\rho^{2h}}{1-\rho}
\end{align*}
\]

hence we achieve the desired result and find an expression for the
variance of the forecast error.

\begin{enumerate}
\def\labelenumi{\arabic{enumi}.}
\setcounter{enumi}{3}
\tightlist
\item
  Show \(\lim_{h\rightarrow \infty} \sigma_h^2 = V(x_t^*)\) and add
  confidence bands to the plot.
\end{enumerate}

\[
\begin{align*}
\sigma^2 \frac{1-\rho^{2h}}{1-\rho} \overset{h\rightarrow \infty}{=} \frac{\sigma^2}{1-\rho}= V(x_t^*)
\end{align*}
\]

\begin{Shaded}
\begin{Highlighting}[]
\FunctionTok{library}\NormalTok{(ggplot2)}

\NormalTok{mu }\OtherTok{\textless{}{-}} \FloatTok{0.2}
\NormalTok{rho }\OtherTok{\textless{}{-}} \FloatTok{0.9}
\NormalTok{sigma }\OtherTok{\textless{}{-}} \FloatTok{1.0}
\NormalTok{T }\OtherTok{\textless{}{-}} \DecValTok{1000}
\NormalTok{h }\OtherTok{\textless{}{-}} \DecValTok{100}

\NormalTok{x }\OtherTok{\textless{}{-}} \FunctionTok{numeric}\NormalTok{(T }\SpecialCharTok{+}\NormalTok{ h)}
\NormalTok{x[}\DecValTok{1}\NormalTok{] }\OtherTok{\textless{}{-}}\NormalTok{ mu }\SpecialCharTok{/}\NormalTok{ (}\DecValTok{1} \SpecialCharTok{{-}}\NormalTok{ rho)}

\FunctionTok{set.seed}\NormalTok{(}\DecValTok{1}\NormalTok{)}
\NormalTok{epsilon }\OtherTok{\textless{}{-}} \FunctionTok{rnorm}\NormalTok{(T }\SpecialCharTok{+}\NormalTok{ h, }\AttributeTok{mean =} \DecValTok{0}\NormalTok{, }\AttributeTok{sd =}\NormalTok{ sigma)}
\ControlFlowTok{for}\NormalTok{ (t }\ControlFlowTok{in} \DecValTok{2}\SpecialCharTok{:}\NormalTok{(T }\SpecialCharTok{+}\NormalTok{ h)) \{}
\NormalTok{  x[t] }\OtherTok{\textless{}{-}}\NormalTok{ mu }\SpecialCharTok{+}\NormalTok{ rho }\SpecialCharTok{*}\NormalTok{ x[t }\SpecialCharTok{{-}} \DecValTok{1}\NormalTok{] }\SpecialCharTok{+}\NormalTok{ epsilon[t]}
\NormalTok{\}}

\NormalTok{forecast\_values }\OtherTok{\textless{}{-}} \FunctionTok{numeric}\NormalTok{(h)}
\NormalTok{lower\_bound }\OtherTok{\textless{}{-}} \FunctionTok{numeric}\NormalTok{(h)}
\NormalTok{upper\_bound }\OtherTok{\textless{}{-}} \FunctionTok{numeric}\NormalTok{(h)}
\NormalTok{z\_alpha }\OtherTok{\textless{}{-}} \FloatTok{1.96}

\ControlFlowTok{for}\NormalTok{ (j }\ControlFlowTok{in} \DecValTok{1}\SpecialCharTok{:}\NormalTok{h) \{}
\NormalTok{  forecast\_values[j] }\OtherTok{\textless{}{-}}\NormalTok{ mu }\SpecialCharTok{*}\NormalTok{ (}\DecValTok{1} \SpecialCharTok{{-}}\NormalTok{ rho}\SpecialCharTok{\^{}}\NormalTok{j) }\SpecialCharTok{/}\NormalTok{ (}\DecValTok{1} \SpecialCharTok{{-}}\NormalTok{ rho) }\SpecialCharTok{+}\NormalTok{ rho}\SpecialCharTok{\^{}}\NormalTok{j }\SpecialCharTok{*}\NormalTok{ x[T]}
\NormalTok{  sigma\_h }\OtherTok{\textless{}{-}}\NormalTok{ sigma }\SpecialCharTok{*} \FunctionTok{sqrt}\NormalTok{((}\DecValTok{1} \SpecialCharTok{{-}}\NormalTok{ rho}\SpecialCharTok{\^{}}\NormalTok{(}\DecValTok{2} \SpecialCharTok{*}\NormalTok{ j)) }\SpecialCharTok{/}\NormalTok{ (}\DecValTok{1} \SpecialCharTok{{-}}\NormalTok{ rho}\SpecialCharTok{\^{}}\DecValTok{2}\NormalTok{))}
\NormalTok{  lower\_bound[j] }\OtherTok{\textless{}{-}}\NormalTok{ forecast\_values[j] }\SpecialCharTok{{-}}\NormalTok{ z\_alpha }\SpecialCharTok{*}\NormalTok{ sigma\_h}
\NormalTok{  upper\_bound[j] }\OtherTok{\textless{}{-}}\NormalTok{ forecast\_values[j] }\SpecialCharTok{+}\NormalTok{ z\_alpha }\SpecialCharTok{*}\NormalTok{ sigma\_h}
\NormalTok{\}}

\NormalTok{forecast\_df }\OtherTok{\textless{}{-}} \FunctionTok{data.frame}\NormalTok{(}
  \AttributeTok{Time =}\NormalTok{ (T }\SpecialCharTok{+} \DecValTok{1}\NormalTok{)}\SpecialCharTok{:}\NormalTok{(T }\SpecialCharTok{+}\NormalTok{ h),}
  \AttributeTok{Forecast =}\NormalTok{ forecast\_values,}
  \AttributeTok{Lower =}\NormalTok{ lower\_bound,}
  \AttributeTok{Upper =}\NormalTok{ upper\_bound}
\NormalTok{)}

\NormalTok{data }\OtherTok{\textless{}{-}} \FunctionTok{data.frame}\NormalTok{(}\AttributeTok{Time =} \DecValTok{1}\SpecialCharTok{:}\NormalTok{(T }\SpecialCharTok{+}\NormalTok{ h), }\AttributeTok{x =}\NormalTok{ x)}

\FunctionTok{ggplot}\NormalTok{(data, }\FunctionTok{aes}\NormalTok{(}\AttributeTok{x =}\NormalTok{ Time, }\AttributeTok{y =}\NormalTok{ x)) }\SpecialCharTok{+}
  \FunctionTok{geom\_line}\NormalTok{(}\AttributeTok{color =} \StringTok{"blue"}\NormalTok{) }\SpecialCharTok{+}
  \FunctionTok{geom\_line}\NormalTok{(}\AttributeTok{data =}\NormalTok{ forecast\_df, }\FunctionTok{aes}\NormalTok{(}\AttributeTok{x =}\NormalTok{ Time, }\AttributeTok{y =}\NormalTok{ Forecast), }\AttributeTok{color =} \StringTok{"red"}\NormalTok{, }\AttributeTok{linetype =} \StringTok{"dashed"}\NormalTok{) }\SpecialCharTok{+}
  \FunctionTok{geom\_ribbon}\NormalTok{(}\AttributeTok{data =}\NormalTok{ forecast\_df, }\FunctionTok{aes}\NormalTok{(}\AttributeTok{x =}\NormalTok{ Time, }\AttributeTok{ymin =}\NormalTok{ Lower, }\AttributeTok{ymax =}\NormalTok{ Upper),}
              \AttributeTok{fill =} \StringTok{"gray"}\NormalTok{, }\AttributeTok{alpha =} \FloatTok{0.3}\NormalTok{, }\AttributeTok{inherit.aes =} \ConstantTok{FALSE}\NormalTok{) }\SpecialCharTok{+}
  \FunctionTok{geom\_hline}\NormalTok{(}\AttributeTok{yintercept =}\NormalTok{ mu }\SpecialCharTok{/}\NormalTok{ (}\DecValTok{1} \SpecialCharTok{{-}}\NormalTok{ rho), }\AttributeTok{linetype =} \StringTok{"dashed"}\NormalTok{, }\AttributeTok{color =} \StringTok{"black"}\NormalTok{) }\SpecialCharTok{+}
  \FunctionTok{labs}\NormalTok{(}\AttributeTok{title =} \StringTok{"Simulated AR(1) Series with Forecast and Confidence Intervals"}\NormalTok{,}
       \AttributeTok{x =} \StringTok{"Time (t)"}\NormalTok{, }\AttributeTok{y =} \FunctionTok{expression}\NormalTok{(x[t])) }\SpecialCharTok{+}
  \FunctionTok{theme\_minimal}\NormalTok{()}
\end{Highlighting}
\end{Shaded}

\includegraphics{FinMetrics-set-4_files/figure-latex/unnamed-chunk-2-1.pdf}

\begin{enumerate}
\def\labelenumi{\arabic{enumi}.}
\setcounter{enumi}{4}
\tightlist
\item
  Try to evaluate how one could use this in practice
\end{enumerate}

One could get some real observed data

\begin{Shaded}
\begin{Highlighting}[]
\FunctionTok{library}\NormalTok{(quantmod)}
\end{Highlighting}
\end{Shaded}

\begin{verbatim}
## Loading required package: xts
\end{verbatim}

\begin{verbatim}
## Loading required package: zoo
\end{verbatim}

\begin{verbatim}
## 
## Attaching package: 'zoo'
\end{verbatim}

\begin{verbatim}
## The following objects are masked from 'package:base':
## 
##     as.Date, as.Date.numeric
\end{verbatim}

\begin{verbatim}
## Loading required package: TTR
\end{verbatim}

\begin{verbatim}
## Registered S3 method overwritten by 'quantmod':
##   method            from
##   as.zoo.data.frame zoo
\end{verbatim}

\begin{Shaded}
\begin{Highlighting}[]
\NormalTok{start }\OtherTok{\textless{}{-}} \DecValTok{2023{-}01{-}01}
\NormalTok{end }\OtherTok{\textless{}{-}} \DecValTok{2024{-}09{-}21}
\NormalTok{ticker }\OtherTok{\textless{}{-}} \StringTok{"F"}
\NormalTok{observed\_data }\OtherTok{\textless{}{-}} \FunctionTok{data.frame}\NormalTok{(}\FunctionTok{getSymbols}\NormalTok{(}\AttributeTok{Symbols =}\NormalTok{ ticker, }\AttributeTok{src =} \StringTok{"yahoo"}\NormalTok{, }\AttributeTok{start =}\NormalTok{ start, }\AttributeTok{end =}\NormalTok{ end, }\AttributeTok{auto.assign =} \ConstantTok{FALSE}\NormalTok{))}
\FunctionTok{head}\NormalTok{(observed\_data)}
\end{Highlighting}
\end{Shaded}

\begin{verbatim}
##            F.Open F.High F.Low F.Close F.Volume F.Adjusted
## 2007-01-03   7.56   7.67  7.44    7.51 78652200   4.205266
## 2007-01-04   7.56   7.72  7.43    7.70 63454900   4.311659
## 2007-01-05   7.72   7.75  7.57    7.62 40562100   4.266862
## 2007-01-08   7.63   7.75  7.62    7.73 48938500   4.328456
## 2007-01-09   7.75   7.86  7.73    7.79 56732200   4.362054
## 2007-01-10   7.79   7.79  7.67    7.73 42397100   4.328456
\end{verbatim}

\begin{Shaded}
\begin{Highlighting}[]
\NormalTok{F\_data }\OtherTok{\textless{}{-}} \FunctionTok{diff}\NormalTok{(}\FunctionTok{log}\NormalTok{(observed\_data}\SpecialCharTok{$}\NormalTok{F.Adjusted))}
\end{Highlighting}
\end{Shaded}

next we fit our model to the observed data.

\begin{Shaded}
\begin{Highlighting}[]
\FunctionTok{library}\NormalTok{(tseries)}

\NormalTok{adf\_test\_result }\OtherTok{\textless{}{-}} \FunctionTok{adf.test}\NormalTok{(F\_data)}
\end{Highlighting}
\end{Shaded}

\begin{verbatim}
## Warning in adf.test(F_data): p-value smaller than printed p-value
\end{verbatim}

\begin{Shaded}
\begin{Highlighting}[]
\FunctionTok{print}\NormalTok{(adf\_test\_result)}
\end{Highlighting}
\end{Shaded}

\begin{verbatim}
## 
##  Augmented Dickey-Fuller Test
## 
## data:  F_data
## Dickey-Fuller = -16.418, Lag order = 16, p-value = 0.01
## alternative hypothesis: stationary
\end{verbatim}

\begin{Shaded}
\begin{Highlighting}[]
\FunctionTok{library}\NormalTok{(forecast)}

\NormalTok{best\_par }\OtherTok{\textless{}{-}} \FunctionTok{auto.arima}\NormalTok{(F\_data)}
\FunctionTok{print}\NormalTok{(best\_par)}
\end{Highlighting}
\end{Shaded}

\begin{verbatim}
## Series: F_data 
## ARIMA(3,0,0) with zero mean 
## 
## Coefficients:
##          ar1     ar2     ar3
##       0.0546  0.0415  0.0277
## s.e.  0.0150  0.0150  0.0150
## 
## sigma^2 = 0.0007292:  log likelihood = 9779.52
## AIC=-19551.05   AICc=-19551.04   BIC=-19525.44
\end{verbatim}

\begin{Shaded}
\begin{Highlighting}[]
\NormalTok{ar1\_model }\OtherTok{\textless{}{-}} \FunctionTok{Arima}\NormalTok{(F\_data, }\AttributeTok{order =} \FunctionTok{c}\NormalTok{(}\DecValTok{3}\NormalTok{,}\DecValTok{0}\NormalTok{,}\DecValTok{0}\NormalTok{), }\AttributeTok{include.mean =} \ConstantTok{FALSE}\NormalTok{)}
\end{Highlighting}
\end{Shaded}

generate a forecast and view it.

\begin{Shaded}
\begin{Highlighting}[]
\NormalTok{h }\OtherTok{\textless{}{-}} \DecValTok{30}

\NormalTok{forecast\_returns }\OtherTok{\textless{}{-}} \FunctionTok{forecast}\NormalTok{(ar1\_model, }\AttributeTok{h =}\NormalTok{ h)}

\FunctionTok{print}\NormalTok{(forecast\_returns)}
\end{Highlighting}
\end{Shaded}

\begin{verbatim}
##      Point Forecast       Lo 80      Hi 80       Lo 95      Hi 95
## 4460  -2.252330e-04 -0.03483076 0.03438030 -0.05314982 0.05269936
## 4461  -3.165376e-04 -0.03497361 0.03434053 -0.05331995 0.05268687
## 4462  -1.283365e-04 -0.03481961 0.03456294 -0.05318406 0.05292739
## 4463  -2.639319e-05 -0.03473579 0.03468301 -0.05310984 0.05305705
## 4464  -1.554260e-05 -0.03472540 0.03469431 -0.05309968 0.05306860
## 4465  -5.501241e-06 -0.03471550 0.03470449 -0.05308986 0.05307885
## 4466  -1.677216e-06 -0.03471170 0.03470835 -0.05308608 0.05308272
## 4467  -7.507540e-07 -0.03471077 0.03470927 -0.05308515 0.05308365
## 4468  -2.630955e-07 -0.03471029 0.03470976 -0.05308466 0.05308414
## 4469  -9.202133e-08 -0.03471012 0.03470993 -0.05308449 0.05308431
## 4470  -3.675536e-08 -0.03471006 0.03470999 -0.05308444 0.05308437
## 4471  -1.311929e-08 -0.03471004 0.03471001 -0.05308441 0.05308439
## 4472  -4.792801e-09 -0.03471003 0.03471002 -0.05308441 0.05308440
## 4473  -1.825081e-09 -0.03471003 0.03471002 -0.05308440 0.05308440
## 4474  -6.622502e-10 -0.03471003 0.03471002 -0.05308440 0.05308440
## 4475  -2.447700e-10 -0.03471002 0.03471002 -0.05308440 0.05308440
## 4476  -9.144355e-11 -0.03471002 0.03471002 -0.05308440 0.05308440
## 4477  -3.351008e-11 -0.03471002 0.03471002 -0.05308440 0.05308440
## 4478  -1.241025e-11 -0.03471002 0.03471002 -0.05308440 0.05308440
## 4479  -4.603316e-12 -0.03471002 0.03471002 -0.05308440 0.05308440
## 4480  -1.695354e-12 -0.03471002 0.03471002 -0.05308440 0.05308440
## 4481  -6.276489e-13 -0.03471002 0.03471002 -0.05308440 0.05308440
## 4482  -2.322426e-13 -0.03471002 0.03471002 -0.05308440 0.05308440
## 4483  -8.572756e-14 -0.03471002 0.03471002 -0.05308440 0.05308440
## 4484  -3.171887e-14 -0.03471002 0.03471002 -0.05308440 0.05308440
## 4485  -1.172792e-14 -0.03471002 0.03471002 -0.05308440 0.05308440
## 4486  -4.333271e-15 -0.03471002 0.03471002 -0.05308440 0.05308440
## 4487  -1.602635e-15 -0.03471002 0.03471002 -0.05308440 0.05308440
## 4488  -5.924631e-16 -0.03471002 0.03471002 -0.05308440 0.05308440
## 4489  -2.189875e-16 -0.03471002 0.03471002 -0.05308440 0.05308440
\end{verbatim}

\begin{Shaded}
\begin{Highlighting}[]
\NormalTok{forecast\_values }\OtherTok{\textless{}{-}}\NormalTok{ forecast\_returns}\SpecialCharTok{$}\NormalTok{mean}
\NormalTok{lower\_bound }\OtherTok{\textless{}{-}}\NormalTok{ forecast\_returns}\SpecialCharTok{$}\NormalTok{lower[,}\DecValTok{2}\NormalTok{]  }
\NormalTok{upper\_bound }\OtherTok{\textless{}{-}}\NormalTok{ forecast\_returns}\SpecialCharTok{$}\NormalTok{upper[,}\DecValTok{2}\NormalTok{]  }

\NormalTok{last\_observed\_price }\OtherTok{\textless{}{-}} \FunctionTok{tail}\NormalTok{(observed\_data}\SpecialCharTok{$}\NormalTok{F.Adjusted, }\DecValTok{1}\NormalTok{)}

\NormalTok{forecast\_prices }\OtherTok{\textless{}{-}}\NormalTok{ last\_observed\_price }\SpecialCharTok{*} \FunctionTok{exp}\NormalTok{(}\FunctionTok{cumsum}\NormalTok{(forecast\_values))}
\NormalTok{lower\_prices }\OtherTok{\textless{}{-}}\NormalTok{ last\_observed\_price }\SpecialCharTok{*} \FunctionTok{exp}\NormalTok{(}\FunctionTok{cumsum}\NormalTok{(lower\_bound))}
\NormalTok{upper\_prices }\OtherTok{\textless{}{-}}\NormalTok{ last\_observed\_price }\SpecialCharTok{*} \FunctionTok{exp}\NormalTok{(}\FunctionTok{cumsum}\NormalTok{(upper\_bound))}

\NormalTok{forecast\_df }\OtherTok{\textless{}{-}} \FunctionTok{data.frame}\NormalTok{(}
  \AttributeTok{Time =} \FunctionTok{seq\_along}\NormalTok{(forecast\_prices) }\SpecialCharTok{+} \FunctionTok{length}\NormalTok{(F\_data),}
  \AttributeTok{Forecast =}\NormalTok{ forecast\_prices,}
  \AttributeTok{Lower =}\NormalTok{ lower\_prices,}
  \AttributeTok{Upper =}\NormalTok{ upper\_prices}
\NormalTok{)}

\NormalTok{observed\_prices }\OtherTok{\textless{}{-}}\NormalTok{ observed\_data}\SpecialCharTok{$}\NormalTok{F.Adjusted[(}\FunctionTok{nrow}\NormalTok{(observed\_data) }\SpecialCharTok{{-}} \DecValTok{59}\NormalTok{)}\SpecialCharTok{:}\FunctionTok{nrow}\NormalTok{(observed\_data)]}
\NormalTok{data\_observed }\OtherTok{\textless{}{-}} \FunctionTok{data.frame}\NormalTok{(}\AttributeTok{Time =}\NormalTok{ (}\FunctionTok{length}\NormalTok{(F\_data) }\SpecialCharTok{{-}} \DecValTok{59}\NormalTok{)}\SpecialCharTok{:}\FunctionTok{length}\NormalTok{(F\_data), }
                            \AttributeTok{Observed =}\NormalTok{ observed\_prices)}

\FunctionTok{ggplot}\NormalTok{(data\_observed, }\FunctionTok{aes}\NormalTok{(}\AttributeTok{x =}\NormalTok{ Time, }\AttributeTok{y =}\NormalTok{ Observed)) }\SpecialCharTok{+}
  \FunctionTok{geom\_line}\NormalTok{(}\AttributeTok{color =} \StringTok{"blue"}\NormalTok{) }\SpecialCharTok{+}
  \FunctionTok{geom\_line}\NormalTok{(}\AttributeTok{data =}\NormalTok{ forecast\_df, }\FunctionTok{aes}\NormalTok{(}\AttributeTok{x =}\NormalTok{ Time, }\AttributeTok{y =}\NormalTok{ Forecast), }\AttributeTok{color =} \StringTok{"red"}\NormalTok{, }\AttributeTok{linetype =} \StringTok{"dashed"}\NormalTok{) }\SpecialCharTok{+}
  \FunctionTok{geom\_ribbon}\NormalTok{(}\AttributeTok{data =}\NormalTok{ forecast\_df, }\FunctionTok{aes}\NormalTok{(}\AttributeTok{x =}\NormalTok{ Time, }\AttributeTok{ymin =}\NormalTok{ Lower, }\AttributeTok{ymax =}\NormalTok{ Upper),}
              \AttributeTok{fill =} \StringTok{"gray"}\NormalTok{, }\AttributeTok{alpha =} \FloatTok{0.3}\NormalTok{, }\AttributeTok{inherit.aes =} \ConstantTok{FALSE}\NormalTok{) }\SpecialCharTok{+}
  \FunctionTok{labs}\NormalTok{(}\AttributeTok{title =} \StringTok{"Forecast with Confidence Intervals on Ford Stock Prices (Last 60 Days)"}\NormalTok{,}
       \AttributeTok{x =} \StringTok{"Time (t)"}\NormalTok{, }\AttributeTok{y =} \StringTok{"Adjusted Close Price"}\NormalTok{) }\SpecialCharTok{+}
  \FunctionTok{theme\_minimal}\NormalTok{()}
\end{Highlighting}
\end{Shaded}

\includegraphics{FinMetrics-set-4_files/figure-latex/unnamed-chunk-7-1.pdf}

This prediction is not very good. We see our CI increase exponentially
over time. One could perform extensive model diagnostics to find better
model.

\end{document}
